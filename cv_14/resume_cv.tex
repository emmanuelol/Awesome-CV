%%%%%%%%%%%%%%%%%%%%%%%%%%%%%%%%%%%%%%%%%
% Awesome Resume/CV
% XeLaTeX Template
% Version 1.1 (9/1/2016)
%
% This template has been downloaded from:
% http://www.LaTeXTemplates.com
%
% Original author:
% Claud D. Park (posquit0.bj@gmail.com) with modifications by 
% Vel (vel@latextemplates.com)
%
% License:
% CC BY-NC-SA 3.0 (http://creativecommons.org/licenses/by-nc-sa/3.0/)
%
% Important note:
% This template must be compiled with XeLaTeX, the below lines will ensure this
%!TEX TS-program = xelatex
%!TEX encoding = UTF-8 Unicode
%
%%%%%%%%%%%%%%%%%%%%%%%%%%%%%%%%%%%%%%%%%

%----------------------------------------------------------------------------------------
%	PACKAGES AND OTHER DOCUMENT CONFIGURATIONS
%----------------------------------------------------------------------------------------

\documentclass[11pt, a4paper]{awesome-cv} % A4 paper size by default, use 'letterpaper' for US letter

\geometry{left=2cm, top=1.5cm, right=2cm, bottom=2cm, footskip=.5cm} % Configure page margins with geometry

\fontdir[fonts/] % Specify the location of the included fonts

% Color for highlights
\colorlet{awesome}{awesome-red} % Default colors include: awesome-emerald, awesome-skyblue, awesome-red, awesome-pink, awesome-orange, awesome-nephritis, awesome-concrete, awesome-darknight
%\definecolor{awesome}{HTML}{CA63A8} % Uncomment if you would like to specify your own color

% Colors for text - uncomment and modify
%\definecolor{darktext}{HTML}{414141}
%\definecolor{text}{HTML}{414141}
%\definecolor{graytext}{HTML}{414141}
%\definecolor{lighttext}{HTML}{414141}

\renewcommand{\acvHeaderSocialSep}{\quad\textbar\quad} % If you would like to change the social information separator from a pipe (|) to something else

%----------------------------------------------------------------------------------------
%	PERSONAL INFORMATION
%	Comment any of the lines below if they are not required
%----------------------------------------------------------------------------------------

\name{Ioannis}{Petrousov}
\address{Greece}
\mobile{(+30) 6955006384}

\email{petrousov@gmail.com}
\homepage{petrousov.net}
\github{gpetrousov}
\linkedin{ioannispetrousov}
%\skype{skypeid}
%\stackoverflow{SOid}{SOname}
%\twitter{@twit}

\position{Computer scientist}%{\enskip\cdotp\enskip}Security Expert} % Your expertise/fields
\quote{``Usque ad finem."} % A quote or statement


\makecvfooter{\today}{Ioannis Petrousov~~~·~~~Résumé}{\thepage} % Specify the letter footer with 3 arguments: (<left>, <center>, <right>), leave any of these blank if they are not needed

%----------------------------------------------------------------------------------------

\begin{document}

\makecvheader % Print the header
\vspace{-0.5cm}
\paragraph{I have no problem undertaking any tests, learning new technologies and moving to a different country.\\}
\bigskip
\noindent
%----------------------------------------------------------------------------------------
%	CV/RESUME CONTENT
%	Each section is imported separately, open each file in turn to modify content
%----------------------------------------------------------------------------------------

%----------------------------------------------------------------------------------------
%	SECTION TITLE
%----------------------------------------------------------------------------------------

\cvsection{Experience}

%----------------------------------------------------------------------------------------
%	SECTION CONTENT
%----------------------------------------------------------------------------------------

\begin{cventries}

%------------------------------------------------

\cventry
{System administrator} % Job title
{Betiator gaming technologies} % Organization
{Athens, Greece} % Location
{Aug. 2015 - Oct. 2016} % Date(s)
{ % Description(s) of tasks/responsibilities
\begin{cvitems}
\item {Designed and implemented a cluster for microservices based on Docker, Mesos, Marathon and Zookeeper.\\
Upgraded to DC/OS later.}
\item {Developed multiple programs for service autoscaling, recovery and reconfiguration.}
\item {Configuration and monitoring of Couchbase clusters.}
\item {Automation and continuous development with Jenkins, Python and BASH}
\item {Networking and security: Mikrotik, HAProxy, CentOS}
\item {\bfseries{Technologies:} Docker, DC/OS, J2EE, Wildfly, Redis, NodeJS, Jenkins, Couchbase, Hadoop}
\end{cvitems}
}

%------------------------------------------------    

\cventry
{Linux administrator (intern)} % Job title
{IP Digital} % Organization
{Thessaloniki, Greece} % Location
{Oct. 2014 - Dec. 2014} % Date(s)
{ % Description(s) of tasks/responsibilities
\begin{cvitems}
\item {High availability cluster: proxmox and ovirt.}
\item {Backup and storage on Freenas and zfs.}
\item {Automation and logging with BASH scripts}
\item {\bfseries{Technologies:} ovirt, unraid, freenas, nas4free, zfsguru, nappit, omnios, openindiana, nexenta and proxmox}
\end{cvitems}
}

%------------------------------------------------

\cventry
{Laboratory Assistant (Rewarding Scholarship)} % Job title
{Laboratory of Digital Systems and Computer Architecture} % Organization
{Kozani, Greece} % Location
{Mar. 2014 - Jun. 2014} % Date(s)
{ % Description(s) of tasks/responsibilities
\begin{cvitems}
\item {Supervising and helping students overcome programming and technical difficulties on the courses}
\begin{itemize}
\item {Operating systems}
\item {Systems of parallel and distributed execution}
\item {Digital design}
\end{itemize}
\end{cvitems}
}

%------------------------------------------------

\cventry
{University Librarian (Rewarding Scholarship)} % Job title
{University Of Western Macedonia} % Organization
{Kozani, Greece} % Location
{Mar. 2013 - Jun. 2013} % Date(s)
{ % Description(s) of tasks/responsibilities
\begin{cvitems}
\item {http://lib.uowm.gr/}
\end{cvitems}
}

%------------------------------------------------

\end{cventries}
%----------------------------------------------------------------------------------------
%	SECTION TITLE
%----------------------------------------------------------------------------------------

\cvsection{Education}

%----------------------------------------------------------------------------------------
%	SECTION CONTENT
%----------------------------------------------------------------------------------------

\begin{cventries}

%------------------------------------------------

\cventry
{B.S. in Informatics and Telecommunications Engineering} % Degree
{UOWM (University of Western Macedonia)} % Institution
{Kozani, Greece} % Location
{Oct 2009 - PRESENT} % Date(s)
{ % Description(s) bullet points
\begin{cvitems}
\item {7 conference publications}
\item {1 journal publication}
\item {2 scholarships}
\item {Current GPA 7.48/10.0}
\end{cvitems}
}

%------------------------------------------------

\end{cventries}
%----------------------------------------------------------------------------------------
%	SECTION TITLE
%----------------------------------------------------------------------------------------

\cvsection{Honors \& Awards}

%----------------------------------------------------------------------------------------
%	INTERNATIONAL SUBSECTION
%----------------------------------------------------------------------------------------

\begin{itemize}[before=\color{black}]
\item{Rewarding scholarship - University of Western Macedonia - 2014}
\item{Rewarding scholarship - University of Western Macedonia - 2013}
\end{itemize}

%\end{cvhonors}

\cvsection{Publications}
\begingroup
\renewcommand{\section}[2]{}%
\bibliographystyle{unsrt}{\color{black}
\bibliography{refs}
\nocite{*}

%----------------------------------------------------------------------------------------
%	SECTION TITLE
%----------------------------------------------------------------------------------------

\cvsection{Extracurricular Activity}

%----------------------------------------------------------------------------------------
%	SECTION CONTENT
%----------------------------------------------------------------------------------------

\begin{cventries}

%------------------------------------------------

\cventry
{Volunteer System administrator} % Affiliation/role
{University Of Western Macedonia} % Organization/group
{Kozani} % Location
{Jan. 2011 - Jul. 2012} % Date(s)
{ % Description(s) of experience/contributions/knowledge
\begin{cvitems}
\item {System maintenance: FreeBSD and Ubuntu}
\item {Virtualization: chroot, jails, VirtualBox}
\item {Storage: ZFS, NFS, Samba}
\item {Wrote many automation and monitoring programs (BASH)}
\end{cvitems}
}

%------------------------------------------------

\end{cventries}

%----------------------------------------------------------------------------------------
%	SECTION TITLE
%----------------------------------------------------------------------------------------

\cvsection{Skills}

%----------------------------------------------------------------------------------------
%	SECTION CONTENT
%----------------------------------------------------------------------------------------

\begin{cvskills}

%------------------------------------------------

\cvskill
{Programming} % Category
{Python, BASH, C/C++, LaTeX, Markdown, Matlab} % Skills

%------------------------------------------------

\cvskill
{Web} % Category
{Basic Frontend. Used Django, Flask and static site generators.} % Skills

%------------------------------------------------

\cvskill
{Languages} % Category
{Greek, English, Russian} % Skills

%------------------------------------------------

\cvskill
{Operating systems} % Category
{No problem using any Linux based OS and working with the command line} % Skills

%------------------------------------------------

\cvskill
{Hardware} % Category
{Logisim, Modelsim, Xilinx Vivado, VHDL, GHDL} % Skills

%------------------------------------------------
\end{cvskills}
\cvsection{Languages}
Greek, English, Russian

%----------------------------------------------------------------------------------------
%	SECTION TITLE
%----------------------------------------------------------------------------------------

\cvsection{Presentations}

%----------------------------------------------------------------------------------------
%	SECTION CONTENT
%----------------------------------------------------------------------------------------

\begin{cventries}

%------------------------------------------------
\bfseries{My presentations are hosted online on speakerdeck https://speakerdeck.com/gpetrousov}

%\cventry
%{Presenter for <DEFCON 20th : The way to go to Las Vegas>} % Role
%{6th CodeEngn (Reverse Engineering Conference)} % Event
%{Seoul, S.Korea} % Location
%{Jul. 2012} % Date(s)
%{ % Description(s)
%\begin{cvitems}
%\item {Introduced CTF (Capture the Flag) hacking competition and advanced techniques and strategy for CTF}
%\end{cvitems}
%}

%------------------------------------------------

\end{cventries}

%%----------------------------------------------------------------------------------------
%	SECTION TITLE
%----------------------------------------------------------------------------------------

\cvsection{Writing}

%----------------------------------------------------------------------------------------
%	SECTION CONTENT
%----------------------------------------------------------------------------------------

\begin{cventries}

%------------------------------------------------

\cventry
{Founder \& Writer} % Role
{A Guide for Developers in Start-up} % Title
{Facebook Page} % Location
{Jan. 2015 - PRESENT} % Date(s)
{ % Description(s)
\begin{cvitems}
\item {Drafted daily news for developers in Korea about IT technologies, issues about start-up.}
\end{cvitems}
}

%------------------------------------------------

\cventry
{Undergraduate Student Reporter} % Role
{AhnLab} % Title
{S.Korea} % Location
{Oct. 2012 - Jul. 2013} % Date(s)
{ % Description(s)
\begin{cvitems}
\item {Drafted reports about IT trends and Security issues on AhnLab Company magazine.}
\end{cvitems}
}

%------------------------------------------------

\end{cventries}
%%----------------------------------------------------------------------------------------
%	SECTION TITLE
%----------------------------------------------------------------------------------------

\cvsection{Program Committees}

%----------------------------------------------------------------------------------------
%	SECTION CONTENT
%----------------------------------------------------------------------------------------

\begin{cvhonors}

%------------------------------------------------

\cvhonor
{Organizer \& Co-director} % Position
{1st POSTECH Hackathon} % Committee
{S.Korea} % Location
{2013} % Date(s)
    
%------------------------------------------------

\cvhonor
{Staff} % Position
{7th Hacking Camp} % Committee
{S.Korea} % Location
{2012} % Date(s)

%------------------------------------------------

\cvhonor
{Problem Writer} % Position
{1st Hoseo University Teenager Hacking Competition} % Committee
{S.Korea} % Location
{2012} % Date(s)

%------------------------------------------------

\cvhonor
{Staff \& Problem Writer} % Position
{JFF(Just for Fun) Hacking Competition} % Committee
{S.Korea} % Location
{2012} % Date(s)

%------------------------------------------------

\end{cvhonors}

%----------------------------------------------------------------------------------------
%\cvsection{Publications}
%\begin{itemize}[before=\color{black}]
%\item{I. Petrousov, M. Dasygenis, "A CAD tool for custom magnitude comparators". Paper presented at 19th Panhellenic Conference on Informatics (PCI '15), Athens Greece}

%\item{I. Petrousov, M. Dasygenis, "Generating custom bitwidth 4:2 compressors in hardware description language from an online tool". Paper presented at International Conference on
%Modern Circuits and Systems Technologies (MOCAST'15), Thessaloniki, Greece}

%\item{M. Dasygenis, I. Petrousov, "Online generation of constant multiplication accelerators". Paper presented at Panhellenic Conference on Electronics and Telecommunications
%(PACET 2015), Ioannina, Greece.}

%\item{M. Dasygenis, I. Petrousov, "A networking EDA Tool for multi-vector multiplication IP circuits". Paper presented at International Conference on Design and Test of Integrated
%Systems in Nanoscale Era (DTIS 2015), IEEE International Conference, Naples, Italy.}

%\item{M. Dasygenis, I. Petrousov, "A generic moduli selection algorithm for the Residue Number System". Paper presented at International Conference on Design and Test of Integrated
%Systems in Nanoscale Era (DTIS 2015), IEEE International Conference, Naples, Italy.}

%\item{I. Petrousov, M. Dasygenis, "Designing optimized forward residue number systems IP blocks converters from a network interface". Paper presented at Panhellenic Conference on
%Informatics (PCI 2014), Athens, Greece.}

%\item{I. Petrousov, M. Dasygenis, “A unique network EDA tool to create optimized ad hoc binary to residue number system converters”. Paper presented at Power And Timing Modeling,
%Optimization and Simulation (PATMOS 2014) conference, Palma de Mallorca.}
%\end{itemize}


%----------------------------------------------------------------------------------------

\cvsection{Projects}
My public projects are hosted on github.

\setlength\parindent{24pt}\color{black}{\textbf{\underline{redis4mesos Feb 2016}}\\
Standalone Redis container to deploy on Mesos cluster with Marathon and route with
HAProxy. This project was featured on Redis weekly \url{http://redisweekly.com/archive/129.html}}

\setlength\parindent{24pt}\color{black}{\textbf{\underline{HDL Generator Jun 2014}}\\
I actively participate in the design, development, debugging and optimization of an online
HDL generator. \url{http://arch.icte.uowm.gr/hdl/}}

\setlength\parindent{24pt}\color{black}{\textbf{\underline{Meta Open Data - M.O.D. May 2014}}\\
The website is used to present the JSON interface with data.gov.gr and facilitate machine-readable
data sets though our API. \url{http://arch.icte.uowm.gr/mod/}

\setlength\parindent{24pt}\color{black}{\textbf{\underline{roboCIM Mar 2014}}\\
The platform allows you to control the LabVolt 5250 robotic arm on a chessboard and play chess or
solve puzzles. \url{https://github.com/gpetrousov/roboCIM}

\setlength\parindent{24pt}\color{black}{\textbf{\underline{Library Management System Jul 2013}}\\
The purpose of this project was to understand the fundamentals of UML. Our final deliverable was
the best semester project in the department. \url{https://goo.gl/QWZIae}



%----------------------------------------------------------------------------------------

\end{document}